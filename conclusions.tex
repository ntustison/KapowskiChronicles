\section{Discussion}
In the absence of ground truth, we used data inspection, prediction of
demographic variables and reproducibility to evaluate the ANTs
cortical thickness pipeline as it applies to large-scale data.  Each
result complements the other by relying upon different information
provided by the pipeline.  Here, we discuss and contextualize each of
the major evaluation points. 

\subsection{Computation Time and Failure Rate}
Computation time for the registration and segmentation components of
the pipeline are substantial.  It is likely that nearly as reliable
results can be obtained in much less time for many of the subjects in
this study.  However, our interest in
maximizing reliability and quality led us to employ parameters in the
registration, segmentation, and bias correction that are as robust as
possible to differences in head position, the presence of large
deformation between template and target brains and substantial
inhomogeneity or artifact within the image content itself.  Several
subjects (e.g. NKI: 1898228, 1875434)
provide examples of more difficult data from which we are able to
extract meaningful segmentations and registrations despite the presence of a
``garbage-in/garbage-out'' problem.  A subject of future study is
determining an exact cut-off for the inclusion of such data.  We do not
investigate this issue, which has concerned statisticians for over
half a century \cite{Hampel2001}, here. 

\subsection{Reproducibility of Thickness Measurements}
The OASIS dataset and the Kirby dataset allow us to test whether the same
thickness values emerge from T1-weighted
neuroimages collected on the same subject but at different times of
the day or over a time separation within a few weeks.  Given that
T1-weighted images are susceptible to short-term alterations due to
blood flow \cite{Franklin2013,Salgado-Pineda2006,Yamasue2007}, this strategy is not ideal.  However, related
tools have looked at this question. An independent evaluation of the FreeSurfer pipeline shows good
repeatability measurements \cite{jovicich2013}. The authors report
FreeSurfer reproducibility in the range of 1.5 - 5\% depending on the
site and region of the brain.  The CLADA pipeline showed the ability to detect
changes as small as 1 millimeter and showed good agreement with
FreeSurfer \cite{nakamura2011}. Very recently, it was suggested that 3T MRI
consistently overestimates cortical thickness \cite{lusebrink2013}.
Repeatability of thickness estimates in that study were in the range
of 0.2 mm although the study design differs substantially from that used here.
In summary, our results (though computed
with a different cortical parcellation) are competitive with these
methods.  Finally, some users may choose to segment and register
with ANTs and subsequently employ any alternative (e.g. surface-based)
method for thickness estimation.  Further work is needed by
independent authors working on established pipelines (as in \cite{lusebrink2013,jovicich2013}) in order to
better compare surface-based and volume-based thickness reliability
across different populations and age ranges. 

\subsection{BrainAGE} 
We used the ANTs cortical thickness pipeline as input to the BrainAGE
algorithm implemented according to \cite{franke2010}.  The original
BrainAGE algorithm was evaluated on a population 19-86 years of age
with n=650 and produced a mean error of 5 years with a correlation
between the testing subject age and real age of 0.92.  Our analysis,
trained on half of our subjects and tested on the other half, produced
an identical correlation and a mean absolute error of 6.4 years.
While the age range in our study is overall similar, we have over
61 subjects younger than 20 years and several dozen subjects with
probable Alzheimer's Disease.  Our study also draws data from multiple
scan sites (even within the same cohort, such as IXI).  
While the true mechanisms underlying BrainAge are unknown, the
primary driving forces from an image processing perspective are accurate affine registration and gray
matter probability images with tissue-derived information that is
relevant to the subject's age.  Therefore, this sub-study serves as
validation of the findings by \cite{franke2010} as well as the
probabilistic segmentation and affine registration components of our
pipeline.

\subsection{General Linear Model Gender \& Age Prediction} 
We also use a training and testing paradigm to evaluate whether a more
traditional prediction model, general linear modeling, is capable of
predicting age or gender from our data.  The general linear model has
the advantage, in comparison to the relevance vector machine, of being
interpretable.  That is, the models reveal the specific predictors
that are most valuable.  

\subsubsection{Site} 
The site variable does not emerge as an important predictor.  This
suggests that our brain volume measurements (which are the most
dominant predictor in the model) are not affected by data collection
site.  Thickness measurements also enter the model.  However, there
are several possible regions that may be used to improve the
classification accuracy and these particular regions are not
necessarily uniquely predictive.  Several components that encode an
interaction between age and thickness as predictors of age also
consistently enter the model.  

\subsubsection{Age} 
The interpretable linear regression model, based on cortical
thickness, finds a correlation with age of 0.77 and a mean absolute
error of 10 years.  The idea, here, is not to produce the best
possible prediction (that was the purpose of BrainAge) but to identify
whether putative regions associated with aging survive within our
model and also to determine the degree to which data collection site
impacts the prediction.  Indeed, our variable selection process
includes SITE in the model suggesting that our analysis pipeline
may not completely remove the effect of SITE on thickness
measurements.  A second reason that SITE is included is that it does,
indeed, covary with age.  That is, the Kirby and NKI subjects are
overall younger than the OASIS and IXI subjects.  Thus, SITE is a
confounding variable in this dataset.  Despite this issue, regions
previously implicated in aging are included in the model such as
occipital lobe, cingulate cortex, several temporal regions, inferior
frontal gyrus (Broca's area) and pre and post-central gyri.  The
superior and middle frontal gyri, known to be relatively preserved in
aging, are absent from the model, as expected.  Thus, despite a
significant impact of SITE on age prediction, biologically plausible
cortical regions emerge in our traditional regression model for age.  
Regions are in agreement, generally, with a
previous large-scale study with a similar age range
\cite{groves2012}. 

\subsection{Network Small-Worldness}
This result is more exploratory and shows the potential of large
datasets such as this one in generating new descriptive and
quantitative hypotheses about aging and gender.  

\subsection{Brain Constellation Maps}
Finally, there is no substitute for looking at one's data.  However,
few researchers have the resources to support detailed 
inspection of over 1200 subjects with cortical thickness
measurements spread across a three-dimensional volume.  Therefore, we
developed ``constellation maps'' which serve to quickly summarize the
cortical thickness measurements across the entire population.   These
maps are derived from the full subjects by cortex % ( $n \times p$ ) 
matrix of thickness measurements
across all subjects and all 32 NIREP regions.  Each column (NIREP
region) is normalized in the range 0 to 1 across all subjects.
Because of this processing, the constellation map allows one to
easily identify non-normal (normal in the statistical sense)
distributions of thickness residuals in the population which,
for the data analyzed in this study, proved to be indicative
of processing failure  during the software refinement stage.  
%For instance, if a subject has a very thin cortex due to segmentation failure (e.g. all
%thickness values less than 1mm) this subject's particular star
%in the plot will .  At the same
%time, if the population thickness is consistently estimated, we should
%be able to see, very easily, an age-related decrease in thickness.
%This is clearly visible in Figure~\ref{fig:stars}.  
%To clarify the
%usefulness of this plot, we include a constellation map from a
%previous run of the pipeline with an approximately 3\% failure rate in
%online materials. 

\section{Conclusions}

Imaging biomarkers such as cortical thickness play an 
important role in neuroscience research.  Extremely useful to
researchers are robust software tools for generating such 
biomarkers.  In this work we detailed our open source offering for estimating
cortical thickness directly from T1 images and demonstrated
its utility on a large collection of public brain data from
multiple databases acquired at multiple sites.  To our knowledge
this study constitutes the largest collection of cortical
thickness data processed in a single study.  
We expect that public availability of our tools and extensive tuning on 
the specified cohorts will prove useful to the larger
research community.  

In this work, we only explored a portion of the potentially
interesting investigations possible with these data.  However,
since all these data are publicly available, further work can
be easily pursued by us or even other interested groups.  We 
note that in addition to making the scripts, csv result files,
and cortical thickness maps available to researchers, 
posterior probability images are also offered permitting additional
quantification such as voxel-based morphometry \citep{ashburner2000}.
