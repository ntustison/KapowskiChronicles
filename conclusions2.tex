
Talk about voxel-based advantage of ANTs.





\section{Discussion}
In the absence of ground truth, we used repeatability and prediction of
demographic variables to compare the ANTs and FreeSurfer cortical 
thickness pipelines as it applies to large-scale data.  One very important
issue which was not discussed in this work is quality control for 
ensuring proper pipeline processing.  The time required to go through 
approximately 1200 sets of results ($\times 2$ for both pipelines) is
enormous (not to mention the tedium).  However, the first
author did do this for the brain extraction step to ensure that expected
intermediate results were being achieved for both pipelines.  The only 
major failure for both pipelines was a FreeSurfer brain extraction of
an IXI subject (IXI430-IOP-0990) shown in Figure \ref{fig:ixi430}. 
Although  researchers might quibble over processing minutiae such as the 
inclusion of too much (or not enough) of the meninges, we approached
our evaluation using more objective criteria which concern all those
engaged in this type of research.  We are currently trying to develop methods
to facilitate data inspection for quick quality assurance/quality control.

\begin{figure}[htb]
\includegraphics[width=87.5mm]{IXI430.png}
\caption{A single FreeSurfer brain extraction failure for an IXI subject.  We 
         overlay the ANTs-estimated brain mask for comparison.}
\label{fig:ixi430}
\end{figure}

\subsection{Repeatability of thickness measurements}
The OASIS dataset and the Kirby dataset allow us to test whether the same
thickness values emerge from T1-weighted
neuroimages collected on the same subject but at different times of
the day or over a time separation within a few weeks.  Given that
T1-weighted images are susceptible to short-term alterations due to
blood flow \cite{Franklin2013,Salgado-Pineda2006,Yamasue2007}, this strategy is not ideal.  However, related
tools have looked at this question. An independent evaluation of the FreeSurfer pipeline shows good
repeatability measurements \cite{jovicich2013}. The authors report
FreeSurfer reproducibility in the range of 1.5 - 5\% depending on the
site and region of the brain.  The CLADA pipeline showed the ability to detect
changes as small as 1 millimeter and showed good agreement with
FreeSurfer \cite{nakamura2011}. Very recently, it was suggested that 3T MRI
consistently overestimates cortical thickness \cite{lusebrink2013}.
Repeatability of thickness estimates in that study were in the range
of 0.2 mm although the study design differs substantially from that used here.
In summary, our results (though computed
with a different cortical parcellation) are competitive.  
Finally, some users may choose to segment and register
with ANTs and subsequently employ any alternative (e.g. surface-based)
method for thickness estimation.  Further work is needed by
independent authors working on established pipelines (as in \cite{lusebrink2013,jovicich2013}) in order to
better compare surface-based and volume-based thickness reliability
across different populations and age ranges. 

\subsection{Age and gender prediction} 
Although repeatability between ANTs and FreeSurfer is comparable,
such measures are not as useful in determining the utility of the 
measuring software.  That is the reason we use a training and testing 
paradigm to evaluate how well both frameworks produce measurements 
capable of predicting demographics which are well-known to correlate
with cortical thickness.  Additionally, these demographic measures are
probably some of the easiest and most reliably obtained of all possible
demographic measures used for this type of assessment.  For age prediction,
we used both a linear model (due to its general ubiquity) and a random
forest model (a non-parametric model to contrast with the linear approach)
which showed overall good performance.  The linear  and
random forest models have the advantage of being
interpretable.  That is, the models reveal the specific predictors
that are most valuable which will be explored in future work.  

\subsection{Computation time}
Computation time for the registration and segmentation components of
the ANTs pipeline are substantial.  It is likely that nearly as reliable
results can be obtained in much less time for many of the subjects in
this study.  However, our interest in
maximizing reliability and quality led us to employ parameters in the
registration, segmentation, and bias correction that are as robust as
possible to differences in head position, the presence of large
deformation between template and target brains and substantial
inhomogeneity or artifact within the image content itself.  Several
subjects (e.g. NKI: 1898228, 1875434) provide examples of more difficult 
data from which we are able to
extract meaningful segmentations and registrations despite the presence of a
``garbage-in/garbage-out'' problem.  A subject of future study is
determining an exact cut-off for the inclusion of such data.  We do not
investigate this issue here, which has concerned statisticians for over
half a century \cite{Hampel2001}. 

\section{Conclusions}

Imaging biomarkers such as cortical thickness play an 
important role in neuroscience research.  Extremely useful to
researchers are robust software tools for generating such 
biomarkers.  In this work we detailed our open source offering for estimating
cortical thickness directly from T1 images and demonstrated
its utility on a large collection of public brain data from
multiple databases acquired at multiple sites.  To our knowledge
this study constitutes the largest collection of cortical
thickness data processed in a single study.  
We expect that public availability of our tools and extensive tuning on 
the specified cohorts will prove useful to the larger
research community.   In this work, we only explored a portion of the potentially
interesting investigations possible with these data.  However,
since all these data are publicly available, further work can
be easily pursued by us or even other interested groups.  