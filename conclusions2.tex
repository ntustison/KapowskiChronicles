\section{Discussion}
In the absence of ground truth, we used repeatability and prediction of
demographic variables to compare the ANTs and FreeSurfer cortical 
thickness pipelines.  
%One very important
%issue that was not discussed in this work is quality control for
%ensuring proper pipeline processing.  The time required to go through 
%approximately 1200 sets of results ($\times 2$ for both pipelines) would be
%enormous (not to mention the tedium).  
%However, the first
%author did do this for the brain extraction step to ensure that both pipelines
%were achieving expected intermediate results.  
The only major failure was the FreeSurfer brain extraction of a single IXI subject 
(IXI430-IOP-0990).  Also, three NKI subjects were not processed to completion
with FreeSurfer (1713515, 18755434, and 2674565) and were not included in the analysis.
Although  researchers might quibble over processing minutiae such as the
inclusion of too much (or not enough) of the meninges, we approached
our evaluation using more objective criteria which concern all those
engaged in this type of research.  We are currently trying to develop methods
to facilitate data inspection for quick quality assurance/control.


\subsection{Repeatability of thickness measurements}
The OASIS data set and the MMRR data set allow us to test 
whether the same thickness values emerge from T1-weighted
MRI collected on the same subject but at different times of
the day or over a time separation within a few weeks.  
Although the ANTs cortical thickness pipeline produced similar
repeatability assessments as FreeSurfer in these data, there
are many additional issues to explore with the ANTs-based
framework.  Pre-analysis confounds such as short-term alterations in cortical 
morphology due to the T1-weighted susceptibility to blood flow 
\citep{Franklin2013,Salgado-Pineda2006,Yamasue2007} and
MRI acquisition parameters such as field strength, site, resolution, 
scanner, longitudinal variation in scanner conditions, and pulse sequence \citep{han2006,lusebrink2013,jovicich2013} have been evaluated
with FreeSurfer which has shown good reliability
under various permutations of these conditions.  
Although we did not explicitly investigate the repeatability performance of the 
ANTs framework under such effects, the relatively good performance
on the large and varied data (in terms of site, field strength, scanner,
and acquisition sequence) used in this study provides confidence in 
its robustness to a variety of imaging conditions.

%However, as we mentioned previously, good repeatability
%does not necessarily translate to accurate measurements.  
%A crucial component in estimating cortical thickness is accurate
%gray matter segmentation for which several algorithms are available.
%A recent study compared FreeSurfer, FSL's FAST \citep{zhang2001},
%SPM8 \citep{ashburner2005}, and VBM8 (an extension of SPM8)%
%\footnote{
%http://dbm.neuro.uni-jena.de/vbm/
%}
%in terms of both reliability and accuracy \citep{eggert2012}.  
%Good accuracy was achieved with the latter three methods with 
%Dice coefficients of $>0.93$ for all regions 
%in both simulated and real T1-weighted data.  In contrast, FreeSurfer
%yielded slightly less accurate results  (mean Dice $> 0.88$) on simulated 
%data and even less accurate results for real data (mean Dice $> 0.58$).
%Despite the relatively poor accuracy, FreeSurfer was extremely reliable,
%demonstrating the least variability in calculated gray matter volumes 
%for single-subject, 10 scan data and the lowest average volume
%differences in the OASIS scan-rescan data.

\subsection{Age and gender prediction} 

Although repeatability between ANTs and FreeSurfer is comparable,
such measures are not as useful in determining the utility of the 
measuring software.  That is the reason we used 
a training and testing paradigm to evaluate how well both frameworks produce measurements capable of predicting demographics which are well-known to correlate
with cortical thickness.  Additionally, these demographic measures are
probably some of the easiest and most reliably obtained of all possible
demographic measures used for this type of assessment.  

{\color{blue}{Previous research has used predictive modeling for comparing cortical
thickness algorithms.  For example, in \cite{clarkson2011}, classification
of healthy, semantic dementia, and progressive non-fluent aphasia categories
using regional cortical thickness values was used to determine the predictive
modeling capabilities of different cortical thickness processing protocols in 
101 subjects. However, differential diagnosis of dementia 
\citep{neary2005} is not as straightforward as obtaining a subject's age
or gender and regressing that against cortical thickness; the latter constitute biological
relationships that have been well-studied and reported in the literature.}
}

For age prediction,
we used both a linear model (due to its general ubiquity) and a random
forest model (a non-parametric model to contrast with the linear approach)
which showed overall good performance.  Also, the linear  and
random forest models have the advantage of being
interpretable.  That is, the models reveal the specific predictors
that are most valuable which will be explored in future work.  
Another interesting result from the age prediction study was that 
predictive performance for the linear model degenerated towards the 10\% level, i.e. 
when using approximately 10\% of the total number of subjects for 
training and the remaining 90\% to test model predictive performance 
(see supplementary material).
This translates into approximately 100 subjects being used for training,
raising concerns about the use of smaller cohorts for performance comparisons
with these data.

This study was limited to a cross-sectional investigation thus limiting
extrapolations of ANTs performance to longitudinal data unlike
recent FreeSurfer extensions which accommodate longitudinal data \citep{reuter2012,jovicich2013}.  
Also, some users may choose to segment and register
with ANTs and subsequently employ any alternative (e.g., surface-based)
method for thickness estimation.  Further work is needed by
independent authors working on established pipelines 
to better compare surface-based and volume-based thickness reliability
and accuracy across different populations, age ranges, and with 
longitudinal protocols. 

\subsection{Computation time}
Computation time for the registration and segmentation components of
the ANTs pipeline are substantial {\color{blue}{but are not significantly worse than
those of FreeSurfer}}.  It is likely that nearly as reliable
results can be obtained in much less time for many of the subjects in
this study.  However, our interest in
maximizing reliability and quality led us to employ parameters in the
registration, segmentation, and bias correction that are as robust as
possible to differences in head position, the presence of large
deformations between template and target brains and substantial
inhomogeneity or other artifacts in the image content itself.  
%Several
%subjects (e.g., NKI: 1898228, 1875434) provide examples of more difficult
%data from which we are able to
%extract meaningful segmentations and registrations, despite the presence of a
%``garbage-in/garbage-out'' problem.  A subject of future study is
%determining an exact cut-off for the inclusion of such data.  We do not
%investigate this issue here, which has concerned statisticians for over
%half a century \citep{Hampel2001}. 

\section{Conclusions}

Imaging biomarkers such as cortical thickness play an 
important role in neuroscience research.  Extremely useful to
researchers are robust software tools for generating such 
biomarkers.  In this work we detailed our open source offering for estimating
cortical thickness directly from T1 images and demonstrated
its utility on a large collection of public brain data from
multiple databases acquired at multiple sites.  To our knowledge,
this study constitutes the largest collection of cortical
thickness data processed in a single study.  
We anticipate that public availability of our tools and extensive tuning on
the specified cohorts will prove useful to the larger
research community.   In this work, we only explored a portion of the potentially
interesting investigations possible with these data.
Since all of the data are publicly available, further work can
be easily pursued by us or by other interested groups.


\section*{Appendix}

{\color{blue}{Available resources are listed in Table \ref{table:resources}
with their corresponding addresses.  Examples and
data for all scripts described in the manuscript are 
also available for download.  This should enable interested
researchers to duplicate the results in this work.}}

\begin{table*}
\caption{{\color{blue}{Resources used in this work.}}}
\label{table:resources}
\centering
\begin{tabular*}{0.95\textwidth}{@{\extracolsep{\fill}} l l}
\toprule
\multicolumn{2}{c}{\bf Packages} \\
\midrule
ANTs & http://stnava.github.io/ANTs \\
FreeSurfer & http://surfer.nmr.mgh.harvard.edu  \\
\midrule
\multicolumn{2}{c}{\bf Available scripts and examples} \\
\midrule
{\tt antsBrainExtraction.sh} & https://github.com/ntustison/antsBrainExtractionExample  \\
{\tt antsAtroposN4.sh} & https://github.com/ntustison/antsAtroposN4Example  \\
{\tt antsCorticalThickness.sh} & https://github.com/ntustison/antsCorticalThicknessExample  \\
{\tt antsMultivariateTemplateConstruction.sh} & https://github.com/ntustison/TemplateBuildingExample  \\
{\tt antsMalfLabeling.sh} & https://github.com/ntustison/MalfLabelingExample  \\
Analysis scripts &  https://github.com/ntustison/KapowskiChronicles \\
\midrule
\multicolumn{2}{c}{\bf Public data} \\
\midrule
MindBoggle101 & http://mindboggle.info/data.html  \\
Cohort templates and priors & http://figshare.com/articles/ANTs\_ANTsR\_Brain\_Templates/915436 \\
IXI & http://biomedic.doc.ic.ac.uk/brain-development \\
MMRR & http://www.nitrc.org/projects/multimodal \\
NKI & http://fcon\_1000.projects.nitrc.org \\
OASIS & http://www.oasis-brains.org \\
MICCAI 2012 Workshop on Multi-Atlas Labeling & https://masi.vuse.vanderbilt.edu/workshop2012/index.php \\
\bottomrule
\end{tabular*}
\end{table*}
